\documentclass[journal,12pt,twocolumn]{IEEEtran}
\usepackage{kvmap}
\usepackage{hyperref}
\usepackage{xcolor}

\definecolor{bgcolor}{gray}{0.95}
\let\oldtexttt\texttt
\renewcommand{\texttt}[1]{%
	\colorbox{bgcolor}{\oldtexttt{#1}}}

\title{XNOR Gate in Arduino}
\author{Ahilan R - FWC22090}

\begin{document}
\maketitle
\bigskip
\begin{abstract}
We implement XNOR logic in Arduino by getting the boolean expression from it's K-map for it's truth table by writing our sketch in C. 
\end{abstract}

\section{\textbf{Theory}}
Let $A$ and $B$ be the inputs to the gate, $Y$ be the output. The truth table for XNOR gate is given below.

%\subsection{Truth Table}
\begin{table}[h]
	\centering
	\begin{tabular}[20pt]{|c|c|c|}
		\hline
		\textbf{A}&\textbf{B}&\textbf{Y}\\
		\hline
		0&0&1\\
		\hline
		0&1&0\\
		\hline
		1&0&0\\
		\hline
		1&1&1\\   
		\hline 
	\end{tabular}
\end{table}

%\subsection{K-map}
K-map for the above mentioned truth table:\\
\begin{table}[h]
	\centering
	\begin{kvmap}
		\kvlist{2}{2}{1,0,0,1}{B,A}
	\end{kvmap}
\end{table}

Simplified K-map:
\begin{table}[h]
	\centering
	\begin{kvmap}
		\kvlist{2}{2}{1,0,0,1}{B,A}
		\bundle[color=red,reducespace=1pt]{0}{0}{0}{0}
		\bundle[color=red,reducespace=1pt]{1}{1}{1}{1}
		\bundle[color=green,reducespace=1pt]{1}{0}{1}{0}
		\bundle[color=green,reducespace=1pt]{0}{1}{0}{1}
	\end{kvmap}
\end{table}

%\subsection{Boolean Expression}
Collecting minterms from the kmap, i.e., where $Y=1$, we get the boolean expression in sum of products form for XNOR gate as following.
\begin{equation}
	Y = A^{\prime}B^{\prime}  +  AB
\end{equation}
And for product of sums form, we get maxterms, where $Y=0$, we get the boolean expression as following.
\begin{equation}
	Y = (A^{\prime}+B)(A+B^{\prime})
\end{equation}
\newpage

\section{\textbf{Implementation}}
\bigskip

%\subsection{Components required}
Components Required:
\begin{table}[h]
	\centering
	\begin{tabular}{|c|c|}
		\hline
		\textbf{Components}&\textbf{Qty.}\\
		\hline
		Arduino UNO&1\\
		\hline
		Breadboard&1\\
		\hline
		Jumper wires&4\\
		\hline
	\end{tabular}
\end{table}

%\subsection{Connections}
Connections:
\begin{table}[h]
	\centering
	\begin{tabular}{|c|c|c|}
		\hline
		\textbf{Arduino}&2&3\\
		\hline
		\textbf{Inputs}&A&B\\
		\hline
	\end{tabular}
\end{table}

In addition to the above table, we connect $5V$ and $GND$ pins of Arduino to different bus strips of breadboard to input binary values back to the input pins of Arduino.

\subsection{Sketch}
We set the digital pins D2, D3 as inputs and feed A, B into those. We utilise the builtin LED at D13, by setting D13 as output. So, when $Y=1$, LED glows and when $Y=0$ LED doesn't glow. This sketch in assembly code is given in the below link.

\bigskip
\url{https://github.com/ahilan22/fwc-1/tree/main/avr-gcc/assignment/code/main.c}

\subsection{Software}
We use avr-gcc compilier to convert the C code into AVR code(binary source that can be uploaded to AVR micro-controller).
\bigskip

\begin{enumerate}
	\item Download the source code and the Makefile in the desired location from the link given above.
	\item Build and execute the code by typing \texttt{make} in the appropriate directory.
\end{enumerate}
\bigskip
We've successfully implemented XNOR logic in Arduino.
\end{document}
