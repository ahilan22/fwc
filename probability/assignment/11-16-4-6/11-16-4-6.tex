%\documentclass[class=article, crop=false]{standalone}
\documentclass{article}

\usepackage{amssymb,amsfonts,amsthm,amsmath}
\usepackage{enumitem}
\usepackage{hyperref,xcolor}
\hypersetup{
    colorlinks,
    urlcolor={black}	%black!50!blue
}

\providecommand{\cbrak}[1]{\ensuremath{\left\{#1\right\}}}
\newcommand{\solution}{\noindent \textbf{Solution: }}
%\newcommand{\varsol}{\noindent \textbf{Aliter: }}
\newcommand*{\permcomb}[4][0mu]{{{}^{#3}\mkern#1#2_{#4}}}
%\newcommand*{\perm}[1][-3mu]{\permcomb[#1]{P}}
\newcommand*{\comb}[1][-1mu]{\permcomb[#1]{C}}
\setlist[enumerate]{font=\small\bfseries}
\renewcommand\thefootnote{\textcolor{black}{\arabic{footnote}}}

\begin{document}

\title{NCERT: Class XI}
\author{\Large Ahilan R - FWC22090}
\date{}

\maketitle

\begin{enumerate}[label=16.\arabic{enumi}.\arabic{enumii}]%,ref=\thesection.\theenumi.\theenumi]
\numberwithin{equation}{enumi}
\setcounter{enumi}{3}
\setcounter{enumii}{6}

\item \footnote{Read question numbers as (CHAPTER NUMBER).(EXERCISE NUMBER).(QUESTION NUMBER)}Three letters are dictated to three persons and an envelope is addressed to each of them, the letters are inserted into the envelopes at random so that each envelope contains exactly one letter. Find the probability that at least one letter is in its proper envelope.\\[1ex]
	\solution
		Let the letters be $X = \cbrak{0,1,2}$ and the persons be $Y = \cbrak{0,1,2}$. Let the placement of letters be $P$.
  The possible placements of letters are,
  \begin{align*}
	  P_1 = \cbrak{0,1,2} &, \quad P_2 = \cbrak{0,2,1} \\
	  P_3 = \cbrak{1,0,2} &, \quad P_4 = \cbrak{1,2,0} \\
	  P_5 = \cbrak{2,0,1} &, \quad P_6 = \cbrak{2,1,0} 
  \end{align*}
  Let $Z$ be the nuber of proper placements, then
  \begin{gather}
	  n(Z=1) = \comb{3}{1} \times 1 \times 1 = 3 \\
	  n(Z=3) = 1 \times 1 \times 1 = 1 \\
	  \therefore \; \text{P}_{req} = \cfrac{n(Z=1)+n(Z=3)}{n(S)} = \cfrac{4}{6} %= \cfrac{2}{3} 
  \end{gather}
		
\noindent\fbox{%
    \parbox{\linewidth}{%
   	%\small%
    }%
}

\end{enumerate}
\end{document}
