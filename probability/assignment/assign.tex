\let\negmedspace\undefined
\let\negthickspace\undefined
\documentclass[journal,12pt,twocolumn]{IEEEtran}
%\documentclass[conference]{IEEEtran}
%\IEEEoverridecommandlockouts
% The preceding line is only needed to identify funding in the first footnote. If that is unneeded, please comment it out.
%\usepackage{cite}
\usepackage{amssymb,amsfonts,amsthm,amsmath}
%\usepackage{algorithmic}
%\usepackage{graphicx}
%\usepackage{textcomp}
%\usepackage{xcolor}
%\usepackage{txfonts}
\usepackage{listings}
\usepackage{enumitem}
%\usepackage{mathtools}
%\usepackage{gensymb}
%\usepackage{bm}
%\usepackage{tikz}
%\usepackage{hyperref}
%\usepackage{color}                                            %%
%\usepackage{array}                                            %%
%\usepackage{longtable}                                        %%
%\usepackage{calc}                                             %%
%\usepackage{multirow}                                         %%
%\usepackage{hhline}                                           %%
%\usepackage{ifthen}                                           %%
\usepackage{bigstrut}
\usepackage{colortbl}

%\renewcommand{\baselinestretch}{2}
\renewcommand\thesection{\arabic{section}}
\renewcommand\thesubsection{\thesection.\arabic{subsection}}
\renewcommand\thesubsubsection{\thesubsection.\arabic{subsubsection}}

\renewcommand\thesectiondis{\arabic{section}}
\renewcommand\thesubsectiondis{\thesectiondis.\arabic{subsection}}
\renewcommand\thesubsubsectiondis{\thesubsectiondis.\arabic{subsubsection}}

% correct bad hyphenation here
\hyphenation{op-tical net-works semi-conduc-tor}
\def\inputGnumericTable{}                                 %%

\lstset{
language=tex,
frame=single, 
breaklines=true
}

\begin{document}

\newtheorem{theorem}{Theorem}[section]
\newtheorem{problem}{Problem}
\newtheorem{proposition}{Proposition}[section]
\newtheorem{lemma}{Lemma}[section]
\newtheorem{corollary}[theorem]{Corollary}
\newtheorem{example}{Example}[section]
\newtheorem{definition}[problem]{Definition}
%\newtheorem{thm}{Theorem}[section] 
%\newtheorem{defn}[thm]{Definition}
%\newtheorem{algorithm}{Algorithm}[section]
%\newtheorem{cor}{Corollary}
\newcommand{\BEQA}{\begin{eqnarray}}
\newcommand{\EEQA}{\end{eqnarray}}
\newcommand{\define}{\stackrel{\triangle}{=}}
\newcommand*\circled[1]{\tikz[baseline=(char.base)]{
    \node[shape=circle,draw,inner sep=2pt] (char) {#1};}}

\bibliographystyle{IEEEtran}
%\bibliographystyle{ieeetr}


\providecommand{\mbf}{\mathbf}
\providecommand{\pr}[1]{\ensuremath{\Pr\left(#1\right)}}
\providecommand{\re}[1]{\ensuremath{\text{Re}\left(#1\right)}}
\providecommand{\im}[1]{\ensuremath{\text{Im}\left(#1\right)}}
\providecommand{\qfunc}[1]{\ensuremath{Q\left(#1\right)}}
\providecommand{\sbrak}[1]{\ensuremath{{}\left[#1\right]}}
\providecommand{\lsbrak}[1]{\ensuremath{{}\left[#1\right.}}
\providecommand{\rsbrak}[1]{\ensuremath{{}\left.#1\right]}}
\providecommand{\brak}[1]{\ensuremath{\left(#1\right)}}
\providecommand{\lbrak}[1]{\ensuremath{\left(#1\right.}}
\providecommand{\rbrak}[1]{\ensuremath{\left.#1\right)}}
\providecommand{\cbrak}[1]{\ensuremath{\left\{#1\right\}}}
\providecommand{\lcbrak}[1]{\ensuremath{\left\{#1\right.}}
\providecommand{\rcbrak}[1]{\ensuremath{\left.#1\right\}}}
\theoremstyle{remark}
\newtheorem{rem}{Remark}
\newcommand{\sgn}{\mathop{\mathrm{sgn}}}
\providecommand{\abs}[1]{\left\vert#1\right\vert}
\providecommand{\res}[1]{\Res\displaylimits_{#1}} 
\providecommand{\norm}[1]{\left\lVert#1\right\rVert}
%\providecommand{\norm}[1]{\lVert#1\rVert}
\providecommand{\mtx}[1]{\mathbf{#1}}
\providecommand{\mean}[1]{E\left[ #1 \right]}
\providecommand{\fourier}{\overset{\mathcal{F}}{ \rightleftharpoons}}
%\providecommand{\hilbert}{\overset{\mathcal{H}}{ \rightleftharpoons}}
\providecommand{\system}{\overset{\mathcal{H}}{ \longleftrightarrow}}
	%\newcommand{\solution}[2]{\textbf{Solution:}{#1}}
\newcommand{\solution}{\noindent \textbf{Solution: }}
\newcommand{\cosec}{\,\text{cosec}\,}
\providecommand{\dec}[2]{\ensuremath{\overset{#1}{\underset{#2}{\gtrless}}}}
\newcommand{\myvec}[1]{\ensuremath{\begin{pmatrix}#1\end{pmatrix}}}
\newcommand{\mydet}[1]{\ensuremath{\begin{vmatrix}#1\end{vmatrix}}}
	\newcommand*{\permcomb}[4][0mu]{{{}^{#3}\mkern#1#2_{#4}}}
\newcommand*{\perm}[1][-3mu]{\permcomb[#1]{P}}
\newcommand*{\comb}[1][-1mu]{\permcomb[#1]{C}}
\newcommand\numberthis{\addtocounter{equation}{1}\tag{\theequation}}
\newcommand\T{\rule{0pt}{2.6ex}}       % Top strut
\newcommand\B{\rule[-1.2ex]{0pt}{0pt}} % Bottom strut
\setlist[enumerate]{font=\small\bfseries}

%\numberwithin{equation}{section}
\numberwithin{equation}{subsection}
%\numberwithin{problem}{section}
%\numberwithin{definition}{section}
\makeatletter
\@addtoreset{figure}{problem}
\makeatother

\let\StandardTheFigure\thefigure
\let\vec\mathbf
\let\j\jmath
%\renewcommand{\thefigure}{\theproblem.\arabic{figure}}
\renewcommand{\thefigure}{\theproblem}
%\setlist[enumerate,1]{before=\renewcommand\theequation{\theenumi.\arabic{equation}}
%\counterwithin{equation}{enumi}


%\renewcommand{\theequation}{\arabic{subsection}.\arabic{equation}}

%%%%%%%%%%%%%%%%%%%%%%%%%%%%%%%%%%%%%%%%%%%%%%%%%%%%%%%%%%%%%%%%%

\title{Probability(NCERT)}
\author{Ahilan R - FWC22090
	%\thanks{Read question numbers as (Chapter-Number).(Exercise-Number).(Question-Number)}
	}

\maketitle

%\tableofcontents
%\begin{abstract}
%	Read question numbers as (Chapter-Number).(Exercise-Number).(Question-Number)
%\end{abstract}

%\begin{abstract}
%Read question numbers as (Chapter-Number).(Exercise-Number).(Question-Number)
%\end{abstract}
%\footnote{Read question numbers as (Chapter-Number).(Exercise-Number).(Question-Number)}

\renewcommand{\thefigure}{\theenumi}
\renewcommand{\thetable}{\theenumi}
\renewcommand{\theequation}{\theenumi}

\section*{Class XI}
%\addcontentsline{toc}{section}{\protect\numberline{}Class XI}
%\section{Class XI}
%\begin{enumerate}[label=\thesection.\arabic*.,ref=\thesection.\theenumi]
\begin{enumerate}[label=16.\arabic{enumi}.\arabic{enumii}]%,ref=\thesection.\theenumi.\theenumi]
\numberwithin{equation}{enumi}
\numberwithin{figure}{enumi}
\numberwithin{table}{enumi}
\setcounter{enumi}{3}
\setcounter{enumii}{6}
\item Three letters are dictated to three persons and an envelope is addressed to each of them, the letters are inserted into the envelopes at random so that each envelope contains exactly one letter. Find the probability that at least one letter is in its proper envelope.\\
	\solution 
	Let the three letter be A, B, and C and the proper order for ABC, i.e., letters to be kept in respective(proper) envelopes. Total number of ways in which letters A, B, C can be inserted into envelopes is \[3!=6=n(S) \numberthis \] Let $X$ be the event where at least one letter is in proper envelope. \[X = \cbrak{\text{ACB, CBA, BAC, ABC}}\] $\implies n(X)=4.$ Therefore, 
  \begin{align}
	  \text{P}(X) &= \cfrac{n(X)}{n(S)}= \cfrac{4}{6} = \cfrac{2}{3} 
  \end{align}
\end{enumerate}

\section*{Class XII}
%\begin{enumerate}[label=\thesection.\arabic*.,ref=\thesection.\theenumi]
\begin{enumerate}[label=13.\arabic{enumi}.\arabic{enumii}]%,ref=\thesection.\theenumi.\theenumi]
\numberwithin{equation}{enumi}
\numberwithin{figure}{enumi}
\numberwithin{table}{enumi}

%%%%% 13.2.3

\setcounter{enumi}{1}
\setcounter{enumii}{3}
\item A box of oranges is inspected by examining three randomly selected oranges drawn without replacement. If all the three oranges are good, the box is approved for sale, otherwise, it is rejected. Find the probability that a box containing 15 oranges out of which 12 are good and 3 are bad ones will be approved for sale.\\
	\solution
	\begin{table}[h!]
	\small
	\centering
		\begin{tabular}{|c|c|c|} \hline
			\textbf{Event}&\textbf{Description}\\ \hline
			$X_1$&Selecting 1 good orange  \\ \hline 
			$X_2$&Selecting 2 good oranges \\ \hline
			$X_3$&Selecting 3 good oranges \\ \hline
		\end{tabular}
	\end{table}
	\begin{table}[h!]
	\small
	\centering
		\begin{tabular}[20pt]{|c|c|c|} \hline
			\textbf{Probability}&\textbf{Value}\\ \hline
			P$(X_1)$ \T  &$\cfrac{12}{15}$  \\[1.5ex] \hline
			P$(X_2|X_1)$&$\cfrac{11}{14}$  \\[1.5ex] \hline
			P$(X_3|X_2X_1)$&$\cfrac{10}{13}$  \\[1.5ex] \hline
		\end{tabular}
	\end{table}\\
		Probability of the given box being approved for sale, (using multiplication rule)
	\begin{align}
		=& {}\text{P}(X_1 X_2 X_3) \\
		=& {}\text{P}(X_1)\text{P}(X_2|X_1)\text{P}(X_3|X_2X_1) \\
		=& \cfrac{12 \times 11 \times 10}{15 \times 14 \times 13} = \cfrac{44}{91}
	\end{align}

%%%%% 13.4.6

\setcounter{enumi}{3}
\setcounter{enumii}{6}
\item From a lot of 30 bulbs which include 6 defectives, a sample of 4 bulbs is drawn at random with replacement. Find the probability distribution of the number of defective bulbs.\\
	\solution
		Let $X$ be the random variable denoting the number of defective bulbs. Let $E_1 \text{ and } E_2$ be the event of drawing a non-defective bulb and a defective bulb.
		\[ \text{P}(E_1)=\cfrac{24}{30} \text{ , } \text{P}(E_2)=\cfrac{6}{30} \numberthis\]
		Since we replace the item after drawing out of the box, (using multiplication rule)
%	\begin{align}
%		\text{P}(X=0) = \cap X_2\cap X_3) \\
%		\text{P}(X_1)\text{P}(X_2|X_1)\text{P}(X_3|X_2X_1) \\
%	\end{align}
	\begin{table}[h]
	\small
	\centering
		\begin{tabular}[20pt]{|c|c|c|} \hline
			\T \B \textbf{Probability}&\textbf{Calculation}&\textbf{Value}\\ \hline
			P$(X=0)$ \T \B &$\comb{4}{0} \times \cfrac{24}{30}\times \cfrac{24}{30}\times \cfrac{24}{30}\times \cfrac{24}{30}$ &$\cfrac{256}{625}$ \\[1.5ex] \hline
			P$(X=1)$ \T \B &$\comb{4}{1} \times \cfrac{6}{30}\times \cfrac{24}{30}\times \cfrac{24}{30}\times \cfrac{24}{30}$  &$\cfrac{256}{625}$ \\[1.5ex] \hline
			P$(X=2)$ \T \B &$\comb{4}{2} \times \cfrac{6}{30}\times \cfrac{6}{30}\times \cfrac{24}{30}\times \cfrac{24}{30}$   &$\cfrac{96}{625}$  \\[1.5ex] \hline
			P$(X=3)$ \T \B &$\comb{4}{3} \times \cfrac{6}{30}\times \cfrac{6}{30}\times \cfrac{6}{30}\times \cfrac{24}{30}$    &$\cfrac{16}{625}$  \\[1.5ex] \hline
			P$(X=4)$ \T \B &$\comb{4}{4} \times \cfrac{6}{30}\times \cfrac{6}{30}\times \cfrac{6}{30}\times \cfrac{6}{30}$     &$\cfrac{1}{625}$   \\[1.5ex] \hline
		\end{tabular}
	\end{table} 
	\begin{table}[h!]
	\normalsize
	\centering
		\newcolumntype{a}{>{\columncolor[gray]{0.95}}c}
		%\begin{tabular}[20pt]{|>{\columncolor[gray]{0.8}}c|>{\columncolor[gray]{0.95}}c|>{\columncolor[gray]{0.95}}c|>{\columncolor[gray]{0.95}}c|>{\columncolor[gray]{0.95}}c|>{\columncolor[gray]{0.95}}c|} \hline
			\begin{tabular}[20pt]{|>{\columncolor[gray]{0.8}}c|a|a|a|a|a|} \hline
			$X$&0&1&2&3&4 \T \\ \hline
			P($X$)&$\cfrac{256}{625}$&$\cfrac{256}{625}$&$\cfrac{96}{625}$&$\cfrac{16}{625}$&$\cfrac{1}{625}$\\[1.5ex] \hline
		\end{tabular}\\[2ex]
		\caption{Probability Distribution of $X$}
	\end{table}\\
\end{enumerate}
\footnotesize{**Read question numbers as (CHAPTER NUMBER).(EXERCISE NUMBER).(QUESTION NUMBER)}
\end{document}
