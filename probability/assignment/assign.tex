\documentclass[journal,12pt,twocolumn]{IEEEtran}
\usepackage{amssymb,amsfonts,amsthm,amsmath}
\usepackage{enumitem}
\usepackage{colortbl}
\usepackage{hyperref,xcolor}
\hypersetup{
    colorlinks,
    urlcolor={black}	%black!50!blue
}

\providecommand{\brak}[1]{\ensuremath{\left(#1\right)}}
\providecommand{\cbrak}[1]{\ensuremath{\left\{#1\right\}}}
\newcommand{\solution}{\noindent \textbf{Solution: }}
%\newcommand{\varsol}{\noindent \textbf{Aliter: }}
\newcommand*{\permcomb}[4][0mu]{{{}^{#3}\mkern#1#2_{#4}}}
\newcommand*{\perm}[1][-3mu]{\permcomb[#1]{P}}
\newcommand*{\comb}[1][-1mu]{\permcomb[#1]{C}}
\newcommand\numberthis{\addtocounter{equation}{1}\tag{\theequation}}
\newcommand\T{\rule{0pt}{2.6ex}}       % Top strut
\newcommand\B{\rule[-1.2ex]{0pt}{0pt}} % Bottom strut
\setlist[enumerate]{font=\small\bfseries}

%\newenvironment{soln}
%  {\renewcommand\qedsymbol{$\blacksquare$}\begin{proof}[Solution]}
%  {\end{proof}}

\begin{document}

\title{Probability(NCERT)}
\author{\Large Ahilan R - FWC22090}
%\renewcommand\UrlFont{\rmfamily\itshape}

\maketitle

\section*{Class XI}
%\addcontentsline{toc}{section}{\protect\numberline{}Class XI}

\begin{enumerate}[label=16.\arabic{enumi}.\arabic{enumii}]%,ref=\thesection.\theenumi.\theenumi]
\numberwithin{equation}{enumi}
\setcounter{enumi}{3}
\setcounter{enumii}{6}
\item \footnote{Read question numbers as (CHAPTER NUMBER).(EXERCISE NUMBER).(QUESTION NUMBER)}Three letters are dictated to three persons and an envelope is addressed to each of them, the letters are inserted into the envelopes at random so that each envelope contains exactly one letter. Find the probability that at least one letter is in its proper envelope.\\
	\solution
		Let the letters be $X = \cbrak{0,1,2}$ and the persons be $Y = \cbrak{0,1,2}$. Let the placement of letters be $P$.
		%\[P = X \times Y = \cbrak{(x_1,0),(x_2,1),(x_3,2): x \in X} \] 
  The possible placements of letters, %neglecting the constant $Y$ elements, %as generated by the code are
  \begin{align*}
	  P_1 = \cbrak{0,1,2} &, \quad P_2 = \cbrak{0,2,1} \\
	  P_3 = \cbrak{1,0,2} &, \quad P_4 = \cbrak{1,2,0} \\
	  P_5 = \cbrak{2,0,1} &, \quad P_6 = \cbrak{2,1,0} 
  \end{align*}
		The placements $P_1, P_2, P_3, P_6$ are favourable outcomes . % The selection of these favorable placements is done through code.\\[1.5ex]
		Hence, the probability =  $\dfrac{4}{6}$.\\ %$0.6667$.\\% =
		
\noindent\fbox{%
    \parbox{\linewidth}{%
   	%\small%
    }%
}
\end{enumerate}

\section*{Class XII}

\begin{enumerate}[label=13.\arabic{enumi}.\arabic{enumii}]
\numberwithin{equation}{enumi}
\numberwithin{table}{enumi}

%%%%% 13.2.3
\setcounter{enumi}{1}
\setcounter{enumii}{3}
\item A box of oranges is inspected by examining three randomly selected oranges drawn without replacement. If all the three oranges are good, the box is approved for sale, otherwise, it is rejected. Find the probability that a box containing 15 oranges out of which 12 are good and 3 are bad ones will be approved for sale.\\
	\solution
		Let the selected oranges be $X = \cbrak{0,1,2}$ and the quality of orange be $Y = \cbrak{0,1}$, where 0 means bad and 1 means good. The desired set of selection is \cbrak{01,11,21}.
	\begin{table}[h!]
	\small
	\centering
		\begin{tabular}[20pt]{|c|c|c|} \hline
			\textbf{Probability}&\textbf{Value}\\ \hline
			P$(01)$ \B \T &$\frac{12}{15}$  \\ \hline
			P$(01|11)$ \B \T &$\frac{11}{14}$  \\ \hline
			P$(01|11 \, 21)$ \T \B &$\frac{10}{13}$  \\ \hline
		\end{tabular}
	\end{table}
		\\ Probability of the given box being approved for sale, (using multiplication rule)
	\begin{align}
		=& \text{ }\text{P}(01 \: 11 \: 21) \\
		=& \text{ }\text{P}(01)\text{P}(01|11)\text{P}(01|11 \: 21) \\
		=& \text{ }\cfrac{12 \times 11 \times 10}{15 \times 14 \times 13} = \cfrac{44}{91}
	\end{align}

%%%%% 13.4.6
\setcounter{enumi}{3}
\setcounter{enumii}{6}
\item From a lot of 30 bulbs which include 6 defectives, a sample of 4 bulbs is drawn at random with replacement. Find the probability distribution of the number of defective bulbs.\\
	\solution
		Let $X$ = number of defective bulbs in a draw. Let $E_1 \text{ and } E_2$ be the event of drawing a non-defective bulb and a defective bulb.
		\[\text{P}(E_2)= p = \dfrac{6}{30}, \quad \text{P}(E_1)= q = \dfrac{24}{30}\]
		\[p+q=1 \implies \text{Bernoulli trials}\]
		Hence, we can define our probability distribution as binomial distribution, B$\brak{4,\frac{6}{30}}$. It's probability function is \[\text{P($X=x$)}=\comb{4}{x}p^xq^{4-x} \numberthis \]
	\begin{table}[h!]
	\normalsize
	\centering
	\newcolumntype{a}{>{\columncolor[gray]{0.95}}c}
		\begin{tabular}[20pt]{|>{\columncolor[gray]{0.8}}c|a|a|a|a|a|} \hline
			$X$&0&1&2&3&4 \T \\ \hline
			P($X$)&$\cfrac{256}{625}$&$\cfrac{256}{625}$&$\cfrac{96}{625}$&$\cfrac{16}{625}$&$\cfrac{1}{625}$\\[1.5ex] \hline
		\end{tabular}\\[2ex]
		\caption{Probability Distribution of $X$}
	\end{table}

\noindent\fbox{%
    \parbox{\linewidth}{%
   	%\small%
    }%
}

\end{enumerate}

\end{document}
