\documentclass[journal,12pt,twocolumn]{IEEEtran}
\usepackage[cmex10]{amsmath}
\usepackage{graphicx}
\usepackage{pgfplots}
\usepackage[font=normal]{caption}
\pgfplotsset{compat=newest}
\pgfplotsset{scaled y ticks=false}
\usepgfplotslibrary{groupplots}
\usepgfplotslibrary{dateplot}

\usepackage{tikz}

\pgfplotsset{compat=1.11,
 /pgfplots/ybar legend/.style={
 /pgfplots/legend image code/.code={
 \draw[##1,/tikz/.cd,yshift=-0.25em]
 (0cm,0cm) rectangle (3pt,0.8em);},
 },
}

\let\vec\mathbf
\newcommand{\myvec}[1]{\ensuremath{\begin{pmatrix}#1\end{pmatrix}}}
\providecommand{\brak}[1]{\ensuremath{\left(#1\right)}}
\providecommand{\norm}[1]{\left\lVert#1\right\rVert}
%\providecommand{\norm}[1]{\ensuremath{\left\lVert(#1\right\rVert)}}
\newcommand\numberthis{\addtocounter{equation}{1}\tag{\theequation}}

\title{Circles using Python}
\author{Ahilan R - FWC22090}
\date{\today}

\begin{document}
\maketitle

\subsection*{\textbf{Problem}}
Let $C_1$ and $C_2$ be two circles with $C_2$ lying inside $C_1$. A circle $C$ lying inside $C_1$ touches $C_1$ internally and $C_2$ externally. Identify the locus of the center of $C$. 

\subsection*{\textbf{Solution}}

\begin{figure}[h]
\centering
\def\figwidth{\linewidth}
\def\figheight{0.35\textheight} % Feel free to change
% This file was created with tikzplotlib v0.10.1.
\begin{tikzpicture}

\definecolor{darkgray176}{RGB}{176,176,176}
\definecolor{gray}{RGB}{128,128,128}
\definecolor{green}{RGB}{0,128,0}
\definecolor{lightgray204}{RGB}{204,204,204}
\definecolor{orange}{RGB}{255,165,0}
\definecolor{steelblue31119180}{RGB}{31,119,180}
\tikzstyle{every node}=[font=\Large]

\begin{axis}[
legend style={nodes={scale=0.75, font=\normalsize, transform shape}, at={(1,0)}, anchor=south east, draw=black},
height=\figheight,
hide x axis,
hide y axis,
legend cell align={left},
legend style={fill opacity=0.8, draw opacity=1, text opacity=1, draw=lightgray204},
tick align=outside,
tick pos=left,
width=\figwidth,
x grid style={darkgray176},
xmin=-17.6, xmax=17.6,
xtick style={color=black},
y grid style={darkgray176},
ymin=-17.5909574051321, ymax=17.5909574051321,
ytick style={color=black}
]
\addplot [draw=steelblue31119180, fill=steelblue31119180, forget plot, mark=*, only marks]
table{%
x  y
0 0
11 0
-4 0
-16 0
8 0
16 0
};
\addplot [semithick, black]
table {%
16 0
15.8686402211719 2.0460345869521
15.4767178086245 4.05847334255212
14.8306681175363 6.00427207806999
13.9410992659742 7.85148083206301
12.8226179498873 9.56976848785945
11.4935896015636 11.1309208096558
9.97583682973974 12.5093037194885
8.2942810929684 13.6822842080855
6.4765334899583 14.630601968253
4.55244138609652 15.3386856485866
2.55359832053407 15.7949085346312
0.512825241146485 15.991779459211
-1.5363684145229 15.9260658071872
-3.56033494330103 15.5988465949092
-5.52584087074092 15.0154947527962
-7.40061264385336 14.185588901968
-9.15386656195471 13.1227560735513
-10.7568142441811 11.8444479532051
-12.1831353339061 10.3716543249246
-13.4094096782694 8.72855841936878
-14.4155018864387 6.94213982588093
-15.1848919521707 5.04173148837793
-15.704946511857 3.05853805922196
-15.9671262840054 1.02512351969141
-15.9671262840054 -1.0251235196914
-15.704946511857 -3.05853805922195
-15.1848919521707 -5.04173148837792
-14.4155018864387 -6.94213982588093
-13.4094096782695 -8.72855841936878
-12.1831353339062 -10.3716543249246
-10.7568142441811 -11.844447953205
-9.15386656195472 -13.1227560735513
-7.40061264385337 -14.185588901968
-5.52584087074093 -15.0154947527962
-3.56033494330103 -15.5988465949092
-1.53636841452292 -15.9260658071872
0.512825241146478 -15.991779459211
2.55359832053405 -15.7949085346312
4.55244138609651 -15.3386856485866
6.47653348995829 -14.630601968253
8.29428109296839 -13.6822842080855
9.97583682973973 -12.5093037194885
11.4935896015636 -11.1309208096558
12.8226179498873 -9.56976848785946
13.9410992659742 -7.85148083206302
14.8306681175363 -6.00427207806999
15.4767178086245 -4.05847334255213
15.8686402211719 -2.04603458695211
16 -3.91886975727153e-15
};
\addlegendentry{$C_1$}
\addplot [semithick, green]
table {%
14 0
13.9753700414697 0.383631485053518
13.9018845891171 0.760963751728522
13.7807502720381 1.12580101463812
13.6139561123702 1.47215265601181
13.4042408656039 1.79433159147365
13.1550480502932 2.08704765181046
12.8704694055762 2.34549444740409
12.5551777049316 2.56542828901604
12.2143500293672 2.74323786904744
11.8535827598931 2.87600355910998
11.4787996851001 2.96154535024335
11.096154732715 2.99845864860206
10.711930922277 2.98613733884759
10.3324371981311 2.92478373654547
9.96390483673608 2.81540526614928
9.6123851292775 2.659797919119
9.28365001963349 2.46051676379087
8.98309732921605 2.22083399122595
8.7156621248926 1.94468518592337
8.48573568532448 1.63660470363165
8.29709339629274 1.30165121735267
8.15283275896799 0.945324654070863
8.0553225290268 0.573475886104118
8.00616382174899 0.19221065994214
8.00616382174899 -0.192210659942138
8.0553225290268 -0.573475886104116
8.15283275896799 -0.945324654070861
8.29709339629274 -1.30165121735267
8.48573568532448 -1.63660470363165
8.7156621248926 -1.94468518592336
8.98309732921605 -2.22083399122595
9.28365001963349 -2.46051676379087
9.61238512927749 -2.659797919119
9.96390483673608 -2.81540526614928
10.3324371981311 -2.92478373654547
10.711930922277 -2.98613733884759
11.096154732715 -2.99845864860206
11.4787996851001 -2.96154535024335
11.8535827598931 -2.87600355910998
12.2143500293672 -2.74323786904744
12.5551777049316 -2.56542828901604
12.8704694055762 -2.34549444740409
13.1550480502932 -2.08704765181046
13.4042408656039 -1.79433159147365
13.6139561123702 -1.47215265601182
13.7807502720381 -1.12580101463812
13.9018845891171 -0.760963751728525
13.9753700414697 -0.38363148505352
14 -7.34788079488412e-16
};
\addlegendentry{$C_2$}
\addplot [semithick, steelblue31119180]
table {%
8 0
7.90148016587895 1.53452594021407
7.60753835646835 3.04385500691409
7.12300108815226 4.50320405855249
6.45582444948067 5.88861062404725
5.61696346241548 7.17732636589459
4.62019220117273 8.34819060724184
3.4818776223048 9.38197778961636
2.2207108197263 10.2617131560642
0.857400117468728 10.9729514761897
-0.585668960427608 11.5040142364399
-2.08480125959945 11.8461814009734
-3.61538106914014 11.9938345944083
-5.15227631089218 11.9445493553904
-6.67025120747577 11.6991349461819
-8.14438065305569 11.2616210645971
-9.55045948289002 10.639191676476
-10.865399921466 9.84206705516347
-12.0676106831358 8.88333596490379
-13.1373515004296 7.77874074369347
-14.0570572587021 6.54641881452659
-14.811626414829 5.2066048694107
-15.388668964128 3.78129861628345
-15.7787098838928 2.29390354441647
-15.975344713004 0.768842639768559
-15.975344713004 -0.768842639768551
-15.7787098838928 -2.29390354441646
-15.388668964128 -3.78129861628344
-14.811626414829 -5.2066048694107
-14.0570572587021 -6.54641881452658
-13.1373515004296 -7.77874074369346
-12.0676106831358 -8.88333596490378
-10.865399921466 -9.84206705516347
-9.55045948289002 -10.639191676476
-8.1443806530557 -11.2616210645971
-6.67025120747578 -11.6991349461819
-5.15227631089219 -11.9445493553904
-3.61538106914014 -11.9938345944083
-2.08480125959946 -11.8461814009734
-0.585668960427616 -11.5040142364399
0.857400117468716 -10.9729514761898
2.22071081972629 -10.2617131560642
3.4818776223048 -9.38197778961636
4.62019220117272 -8.34819060724184
5.61696346241548 -7.17732636589459
6.45582444948067 -5.88861062404726
7.12300108815226 -4.50320405855249
7.60753835646835 -3.0438550069141
7.90148016587895 -1.53452594021408
8 -2.93915231795365e-15
};
\addlegendentry{$C$}
\addplot [semithick, black, forget plot]
table {%
-16 0
-13 0
-10 0
-7 0
-4 0
-1 0
2 0
5 0
8 0
11 0
};
\addplot [semithick, gray, dashed, forget plot]
table {%
11 0
11.5555555555556 0
12.1111111111111 0
12.6666666666667 0
13.2222222222222 0
13.7777777777778 0
14.3333333333333 0
14.8888888888889 0
15.4444444444444 0
16 0
};
\addplot [semithick, orange, dashed]
table {%
15 0
14.9808734264829 0.491279569162143
14.9235707218926 0.980580928893333
14.8283226239957 1.4659338353138
14.6955126632654 1.94538394357167
14.525675618539 2.41700067729991
14.3194953636527 2.878885002366
14.0778021137229 3.32917707361209
13.801569082163 3.76606372379504
13.4919085618962 4.18778576457099
13.1500674465451 4.5926450701259
12.7774222096303 4.97901141492888
12.3754733619982 5.34532903807499
11.9458394097928 5.69012290778536
11.490250337303 6.01200466083949
11.0105406409264 6.309678193024
10.5086419422998 6.58194487808681
9.98657521034049 6.827708394182
9.44644262351792 7.04597913837094
8.89041910512278 7.2358782114041
8.32074356561861 7.39664095673806
7.73970988733956 7.52762003953758
7.14965768783584 7.62828805326446
6.55296289905961 7.69823964335758
5.95202820032555 7.73719313945261
5.34927334356932 7.74499168956917
4.74712540986051 7.7216038916984
4.14800903640379 7.66712392024773
3.55433665338069 7.58177114683384
2.96849876994467 7.46588925695053
2.39285434848449 7.31994486606857
1.82972130591528 7.14452564073988
1.28136718024515 6.94033793227174
0.749999999999998 6.70820393249937
0.237759393271952 6.4490583631096
-0.253292027807835 6.16394471184655
-0.721176972480207 5.85401103075477
-1.16401143321005 5.52050531337886
-1.58001227191967 5.16477046953393
-1.96750440005648 4.78823891788173
-2.32492752358341 4.39242681808631
-2.65084242573228 3.97892796577414
-2.94393676222177 3.54940737488162
-3.20303034560466 3.10559457323168
-3.42707989746613 2.64927663833558
-3.61518324933773 2.18229100146258
-3.76658297541137 1.70651804895271
-3.88066944242575 1.22387355056462
-3.9569832644443 0.736300945346939
-3.99521715264026 0.245763516095185
-3.99521715264026 -0.245763516095186
-3.9569832644443 -0.736300945346941
-3.88066944242575 -1.22387355056462
-3.76658297541137 -1.70651804895271
-3.61518324933773 -2.18229100146258
-3.42707989746613 -2.64927663833558
-3.20303034560466 -3.10559457323168
-2.94393676222177 -3.54940737488162
-2.65084242573228 -3.97892796577414
-2.32492752358341 -4.39242681808631
-1.96750440005648 -4.78823891788173
-1.58001227191967 -5.16477046953392
-1.16401143321005 -5.52050531337886
-0.72117697248021 -5.85401103075477
-0.253292027807833 -6.16394471184655
0.237759393271951 -6.4490583631096
0.750000000000004 -6.70820393249937
1.28136718024515 -6.94033793227174
1.82972130591528 -7.14452564073988
2.39285434848449 -7.31994486606857
2.96849876994467 -7.46588925695053
3.55433665338069 -7.58177114683384
4.14800903640379 -7.66712392024773
4.74712540986051 -7.7216038916984
5.34927334356933 -7.74499168956917
5.95202820032555 -7.73719313945261
6.5529628990596 -7.69823964335758
7.14965768783583 -7.62828805326446
7.73970988733956 -7.52762003953758
8.32074356561861 -7.39664095673806
8.89041910512278 -7.2358782114041
9.44644262351792 -7.04597913837094
9.98657521034049 -6.827708394182
10.5086419422998 -6.58194487808681
11.0105406409264 -6.309678193024
11.490250337303 -6.01200466083949
11.9458394097928 -5.69012290778535
12.3754733619982 -5.34532903807499
12.7774222096303 -4.97901141492888
13.1500674465451 -4.5926450701259
13.4919085618962 -4.18778576457099
13.801569082163 -3.76606372379504
14.0778021137229 -3.32917707361209
14.3194953636527 -2.878885002366
14.525675618539 -2.41700067729991
14.6955126632654 -1.94538394357167
14.8283226239957 -1.46593383531379
14.9235707218926 -0.980580928893331
14.9808734264829 -0.491279569162143
15 -1.89721466323357e-15
};
\addlegendentry{Locus of $P$}
\draw (axis cs:0,0) node[
  scale=0.5,
  anchor=base west,
  text=black,
  rotate=0.0
]{A};
\draw (axis cs:11,0) node[
  scale=0.5,
  anchor=base west,
  text=black,
  rotate=0.0
]{B};
\draw (axis cs:-4,0) node[
  scale=0.5,
  anchor=base west,
  text=black,
  rotate=0.0
]{P};
\draw (axis cs:-16,0) node[
  scale=0.5,
  anchor=base west,
  text=black,
  rotate=0.0
]{$T_1$};
\draw (axis cs:8,0) node[
  scale=0.5,
  anchor=base west,
  text=black,
  rotate=0.0
]{$T_2$};
\draw (axis cs:16,0) node[
  scale=0.5,
  anchor=base west,
  text=black,
  rotate=0.0
]{D};
\end{axis}

\end{tikzpicture}

\label{fig:rough}
\caption{Rough sketch of given problem}
\end{figure}

\begin{table}[h]
	\small
	\centering
	\begin{tabular}[20pt]{|c|c|}
		\hline
		\textbf{Symbol}&\textbf{Description}\\
		\hline
			$A$&Center of $C_1$\\
		\hline
			$B$&Center of $C_2$\\
		\hline
			$P$&Center of $C$\\
		\hline
			$r_1$&Radius of $C_1$\\
		\hline 
			$r_2$&Radius of $C_2$\\
		\hline 
			$r$&Radius of $C$\\
		\hline 
	\end{tabular}
\end{table}

From the diagram we can deduce, 
%\begin{align} TeXLive doesn't allow split inside align?? But why tho?
\begin{equation}
\begin{split}
	AP &= AT_1 - PT_1 \\
	&= r_A - r \label{eq:1}
\end{split} 
\end{equation} 
\begin{equation}
\begin{split}
	BP &= BT_2 + PT_2 \\
	&= r - r_A \label{eq:2}
\end{split} 
\end{equation} 
%\end{align} 

Adding \eqref{eq:1} and \eqref{eq:2} we get,
\begin{equation} \label{eq:const}
\begin{split}
	AP + BP &= (r_A-r) + (r-r_B) \\
	&=r_A+r_B = c 
\end{split}
\end{equation}
%	\text{Let, } \quad c &= r_A + r_B \\
%		\implies \quad AP + BP &= c \label{eq:const}
Since R.H.S of \eqref{eq:const} is constant, we can say that the locus of $P$ is ellipse, by the defintion of ellipse.

\subsection*{\textbf{Construction}}
Let the circles $C_1$, $C_2$ and $C$ be represented by following equations respectively,
\begin{align*}
		\vec{x^{\top}}\vec{I}\vec{x} + 2\vec{u_1^{\top}}\vec{x} + f_2 = 0 \\
		\vec{x^{\top}}\vec{I}\vec{x} + 2\vec{u_2^{\top}}\vec{x} + f_2 = 0 \\
		\vec{x^{\top}}\vec{I}\vec{x} + 2\vec{u^{\top}}\vec{x} + f = 0
\end{align*}
where,
\begin{table}[h]
		\normalsize
	\centering
	\begin{tabular}[20pt]{|c|c|c|}
		\hline
		%			\textbf{Symbol}&\textbf{Description}&\textbf{Value}\\
		%		\hline
			Center of $C_1$&$\vec{A}$&$-\vec{u_1}$\\
		\hline                           
			Center of $C_2$&$\vec{B}$&$\vec{-u_2}$\\
		\hline                           
			Center of $C$&$\vec{P}$&$\vec{-u}$\\
		\hline
	\end{tabular}
%\caption{Defining $u_1$ and $u_2$}
\end{table} \\
For the sake of simplicity, let's assume $\vec{A} = \myvec{0\\0}$.% We can linear transform back to the original form if needed.
\begin{align*}
	AP &= \norm{\vec{A}-\vec{P}} \\
	&= \norm{\vec{u}} \\
	BP &= \norm{\vec{B}-\vec{P}} \\
	&= \norm{\vec{u_2-u}}
\end{align*} 
Now, \eqref{eq:const} becomes 
\begin{gather*}
	\norm{\vec{u_2-u}}+\norm{\vec{u}} = c\\
	\norm{\vec{u_2-u}}^2 = \brak{c-\norm{\vec{u}}}^2 \\
%\begin{align*}
	\brak{\vec{{u_2-u}^{\top}}}\brak{\vec{u_2-u}} = c^2 + \vec{u^{\top}}\vec{u} -2c\norm{\vec{u}} \\
	\vec{u_2^{\top}}\vec{u_2} + \vec{u^{\top}}\vec{u} - 2\vec{u_2^{\top}}\vec{u} = c^2 + \vec{u^{\top}}\vec{u} -2c\norm{\vec{u}} \\
	- 2\vec{u_2^{\top}}\vec{u} = c^2 - \vec{u_2^{\top}}\vec{u_2} - 2c\norm{\vec{u}} 
\end{gather*}
%\end{align*} 
Taking $\vec{u_2^{\top}}\vec{u_2} - c^2 = k$, %and squaring on both sides,
\begin{gather*}
	\brak{k - 2\vec{u_2^{\top}}\vec{u}}^2 =  \brak{-2c\norm{\vec{u}}}^2 \\
	k^2 + 4\brak{\vec{u_2^{\top}}\vec{u}}\brak{\vec{u_2^{\top}}\vec{u}} - 4k\vec{u_2^{\top}}\vec{u} =  4c^2\vec{u^{\top}}\vec{u} \\
	4c^2\vec{u^{\top}}\vec{u} - 4\vec{u^{\top}}\brak{\vec{u_2}\vec{u_2^{\top}}}\vec{u} + 4k\vec{u_2^{\top}}\vec{u} - k^2 =  0 \\
		\implies \quad \vec{x^{\top}}\vec{V}\vec{x} + 2\vec{w^{\top}}\vec{x} + g = 0 \numberthis \label{eq:lips}\\
\end{gather*} 
\eqref{eq:lips} is the equation of the ellipse, the locus of $P$, where,
\begin{align*}
	\vec{V} &= 4c^2\vec{I} - 4\vec{u_2}\vec{u_2^{\top}} \text{,}\\
	\vec{w} &= 2k\vec{u_2} \quad \text{and} \quad g = -k^2 \text{.}
\end{align*}

\subsection*{\textbf{Construction Parameters Used}}

\begin{table}[h!]
		\small
	\centering
	\begin{tabular}[20pt]{|c|c|c|}
		\hline
		\textbf{Symbol}&\textbf{Description}&\textbf{Value}\\
		\hline
			$A$&Center of $C_1$&$\myvec{0\\0}$\\
		\hline
			$B$&Center of $C_2$&$\myvec{11\\0}$\\
		\hline
			$P$&Center of $C$&$\myvec{-4\\0}$\\
		\hline
			$r_1$&Radius of $C_1$&16\\
		\hline 
			$r_2$&Radius of $C_2$&3\\
		\hline 
			$r$&Radius of $C$&12\\
		\hline 
			$r$&Radius of $C$&12\\
		\hline 
			$T_1D$&Diameter of $C_1$&\\
		\hline 
			gap&Length of $BD$&2\\
		\hline 
	\end{tabular}
    \caption{Circles parameters}
	\label{table:1}
\end{table}
On substituting respective values taken from \ref{table:1}, the equation of the locus of, $P$, \eqref{eq:lips} will be

\begin{align}
		\vec{x^{\top}}\myvec{60&0\\0&90.25}\vec{x} + 2\myvec{330&0}\vec{x} - 3600 = 0 \label{eq:eli}
\end{align}

\begin{table}[h!]
	\small
	\centering
	\begin{tabular}[20pt]{|c|c|c|}
		\hline
		\textbf{Symbol}&\textbf{Description}&\textbf{Value}\\
		\hline 
			a&Semi-major axis&9.5\\
		\hline 
			b&Semi-minor axis&7.746\\
		\hline 
			$\vec{c}$&Center of Ellipse&$\myvec{5.5\\0}$\\
		\hline 
	\end{tabular}
		\caption{Ellipse parameters from \eqref{eq:eli}}
	\label{table:2}
\end{table}

\end{document}
