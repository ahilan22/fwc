\documentclass[journal,12pt,twocolumn]{IEEEtran}
\usepackage[cmex10]{amsmath}
\usepackage[short]{optidef}
%\usepackage{bm}
\usepackage{graphicx}
\usepackage[font=normal]{caption}

\usepackage{pgfplots}
\pgfplotsset{compat=newest}
\pgfplotsset{scaled y ticks=false}
\usepgfplotslibrary{groupplots}
\usepgfplotslibrary{dateplot}

\usepackage{tikz}

\pgfplotsset{compat=1.11,
 /pgfplots/ybar legend/.style={
 /pgfplots/legend image code/.code={
 \draw[##1,/tikz/.cd,yshift=-0.25em]
 (0cm,0cm) rectangle (3pt,0.8em);},
 },
}

\let\vec\mathbf
%\newcommand{\myvec}[1]{\ensuremath{\begin{pmatrix}#1\end{pmatrix}}}
%\newcommand{\mydet}[1]{\ensuremath{\begin{vmatrix}#1\end{vmatrix}}}
\providecommand{\brak}[1]{\ensuremath{\left(#1\right)}}
%\providecommand{\norm}[1]{\left\lVert#1\right\rVert}
%\newcommand\numberthis{\addtocounter{equation}{1}\tag{\theequation}}
%\providecommand{\abs}[1]{\left\vert#1\right\vert}
%\providecommand{\lbrak}[1]{\ensuremath{\left(#1\right.}}
%\providecommand{\rbrak}[1]{\ensuremath{\left.#1\right)}}
%\providecommand{\sbrak}[1]{\ensuremath{{}\left[#1\right]}}


\title{Optimization Assignment}
\author{Ahilan R - FWC22090}

\begin{document}
\maketitle

\subsection*{\textbf{Problem}}
Find the maximum area of a triangle which can be inscribed in an ellipse.
%\subsection*{\textbf{Solution}}
\subsection*{\textbf{Using Calculus}}
Let's take an ellipse,
\begin{equation}
		\cfrac{x^2}{a^2} + \cfrac{y^2}{b^2} = 1 \label{eq:ell_og}
\end{equation}
We can write \eqref{eq:ell_og} as
\[	x^2 + \cfrac{y^2}{\left( \frac{b}{a} \right) ^2} = a^2 \]
\begin{equation}
x^2 + Y^2 = a^2 \label{eq:ell_circ}
\end{equation}
Scaling along $y$-axis by a factor of $\frac{b}{a}$, we get new $Y$-axis. Here, \eqref{eq:ell_og} becomes \eqref{eq:ell_circ}, which is a circle with radius $a$. 

\begin{figure}[h]
\centering
\def\figwidth{\linewidth}
\def\figheight{0.35\textheight} % Feel free to change
% This file was created with tikzplotlib v0.10.1.
\begin{tikzpicture}

\definecolor{darkgray176}{RGB}{176,176,176}
\definecolor{steelblue31119180}{RGB}{31,119,180}
\tikzstyle{every node}=[font=\Large]

\begin{axis}[
legend style={nodes={scale=0.75,font= \small, transform shape}, at={(1,0)}, anchor=south east, draw=black},
height=\figheight,
hide x axis,
hide y axis,
tick align=outside,
tick pos=left,
width=\figwidth,
x grid style={darkgray176},
xmin=-5.48921331193927, xmax=5.49948634818758,
xtick style={color=black},
y grid style={darkgray176},
ymin=-5.5, ymax=5.5,
ytick style={color=black}
]
\addplot [draw=steelblue31119180, fill=steelblue31119180, mark=*, only marks]
table{%
x  y
0 0
-4 -3
4 -3
0 5
0 -3
0 -5
};
\addplot [semithick, black]
table {%
-4 -3
-3.11111111111111 -3
-2.22222222222222 -3
-1.33333333333333 -3
-0.444444444444445 -3
0.444444444444445 -3
1.33333333333333 -3
2.22222222222222 -3
3.11111111111111 -3
4 -3
};
\addplot [semithick, black, dotted]
table {%
-4 -3
-3.55555555555556 -2.66666666666667
-3.11111111111111 -2.33333333333333
-2.66666666666667 -2
-2.22222222222222 -1.66666666666667
-1.77777777777778 -1.33333333333333
-1.33333333333333 -1
-0.888888888888889 -0.666666666666667
-0.444444444444445 -0.333333333333333
0 0
};
\addplot [semithick, black]
table {%
4 -3
3.55555555555556 -2.11111111111111
3.11111111111111 -1.22222222222222
2.66666666666667 -0.333333333333333
2.22222222222222 0.555555555555555
1.77777777777778 1.44444444444444
1.33333333333333 2.33333333333333
0.888888888888889 3.22222222222222
0.444444444444445 4.11111111111111
0 5
};
\addplot [semithick, black]
table {%
0 -5
0 -3.88888888888889
0 -2.77777777777778
0 -1.66666666666667
0 -0.555555555555555
0 0.555555555555555
0 1.66666666666667
0 2.77777777777778
0 3.88888888888889
0 5
};
\addplot [semithick, black]
table {%
-4 -3
-3.55555555555556 -2.11111111111111
-3.11111111111111 -1.22222222222222
-2.66666666666667 -0.333333333333333
-2.22222222222222 0.555555555555555
-1.77777777777778 1.44444444444444
-1.33333333333333 2.33333333333333
-0.888888888888889 3.22222222222222
-0.444444444444445 4.11111111111111
0 5
};
\addplot [semithick, black]
table {%
5 0
4.95895006911623 0.63938580842253
4.83647431519515 1.26827291954754
4.63458378673011 1.87633502439687
4.35659352061695 2.45358776001969
4.00706810933978 2.99055265245608
3.59174675048864 3.47841275301743
3.11744900929367 3.90915741234015
2.59196284155263 4.27571381502673
2.02391671561197 4.57206311507906
1.42263793315516 4.7933392651833
0.797999475166897 4.93590891707225
0.160257887858277 4.99743108100344
-0.480115129538408 4.97689556474599
-1.11260466978157 4.87463956090912
-1.72682527210654 4.6923421102488
-2.31269145120418 4.432996531865
-2.86058330061085 4.10086127298478
-3.36150445130658 3.70138998537658
-3.80722979184567 3.24114197653894
-4.1904405244592 2.72767450605274
-4.5048443395121 2.16941869558779
-4.74527873505334 1.5755410901181
-4.90779578495533 0.955793143506863
-4.98972696375168 0.320351099903566
-4.98972696375168 -0.320351099903563
-4.90779578495533 -0.955793143506859
-4.74527873505334 -1.5755410901181
-4.5048443395121 -2.16941869558779
-4.1904405244592 -2.72767450605274
-3.80722979184567 -3.24114197653894
-3.36150445130659 -3.70138998537658
-2.86058330061085 -4.10086127298478
-2.31269145120418 -4.432996531865
-1.72682527210654 -4.6923421102488
-1.11260466978157 -4.87463956090912
-0.480115129538412 -4.97689556474599
0.160257887858274 -4.99743108100344
0.797999475166892 -4.93590891707225
1.42263793315516 -4.7933392651833
2.02391671561196 -4.57206311507906
2.59196284155262 -4.27571381502673
3.11744900929367 -3.90915741234015
3.59174675048864 -3.47841275301743
4.00706810933978 -2.99055265245608
4.35659352061695 -2.45358776001969
4.63458378673011 -1.87633502439687
4.83647431519515 -1.26827291954754
4.95895006911623 -0.639385808422533
5 -1.22464679914735e-15
};
\draw (axis cs:0,0) node[
  scale=0.5,
  anchor=base west,
  text=black,
  rotate=0.0
]{O};
\draw (axis cs:-4,-3) node[
  scale=0.5,
  anchor=base west,
  text=black,
  rotate=0.0
]{A};
\draw (axis cs:4,-3) node[
  scale=0.5,
  anchor=base west,
  text=black,
  rotate=0.0
]{B};
\draw (axis cs:0,5) node[
  scale=0.5,
  anchor=base west,
  text=black,
  rotate=0.0
]{C};
\draw (axis cs:0,-3) node[
  scale=0.5,
  anchor=base west,
  text=black,
  rotate=0.0
]{D};
\draw (axis cs:0,-5) node[
  scale=0.5,
  anchor=base west,
  text=black,
  rotate=0.0
]{E};
\draw (axis cs:0.2,-1.5) node[
  scale=0.5,
  anchor=base west,
  text=black,
  rotate=0.0
]{$(h-a)$};
\draw (axis cs:-1.5,-2.8) node[
  scale=0.5,
  anchor=base west,
  text=black,
  rotate=0.0
]{$b$};
\draw (axis cs:-2.2,-1.3) node[
  scale=0.5,
  anchor=base west,
  text=black,
  rotate=0.0
]{$a$};
\end{axis}

\end{tikzpicture}

\label{fig:roughSketch}
\end{figure}

Let $CE$ be perpendicular to $AB$. Then $CD=h$. $CD$ will pass through the center $O$ of the circle, since $h$ should be maximum for a given chord, for $\triangle ABC$ to have maximum area. Also $AD=DB=b$. Therefore, 
\begin{equation}
R = ar(\triangle ABC)= \frac{1}{2}(2b)h= bh \label{eq:arABC}
\end{equation}
From $\triangle ADO$, we have
\begin{align}
		b^2 =& a^2- (h-a)^2 \nonumber \\ %\[ {AD}^2= {OA}^2- {OD}^2 \] \\
		=& 2ah-h^2 \label{eq:baseb}
\end{align}
Substituting \eqref{eq:baseb} in $R^2$ from \eqref{eq:arABC},
\begin{align}
		R^2 &= (2ah-h^2)h^2 \nonumber \\
		&= 2ah^3-h^4 \\
		\implies f(x) &= 2ax^3-x^4 \label{eq:fofx} \\
		f^{\prime}(x) &= 6ax^2 -4x^3 \label{eq:fprime} \\
		f^{\prime\prime}(x) &= 12ax -12x^2 \label{eq:fdprime}
\end{align}

On solving for maxima of $f(x)$ from \eqref{eq:fprime} and \eqref{eq:fdprime}, we get $x=\frac{3}{2}a$. Upon substitution in \eqref{eq:fofx}, we have
\begin{align}
		R^2 &= \cfrac{27}{16}a^4 \nonumber\\
		\implies R &= \cfrac{3\sqrt{3}}{4}a^2 \label{eq:ar_circ}
\end{align}

On scaling back to the original axes, we get the area of the ellipse \eqref{eq:ell_og}, $S$. \eqref{eq:ar_circ} becomes
\begin{align}
	S &= R \brak{\cfrac{b}{a}} \label{eq:scaleBack} \\
		S &= \cfrac{3\sqrt{3}}{4}ab \label{eq:ar_ellip}
\end{align}

\subsection*{\textbf{Gradient Ascent}}

Since area is a positive quantity, $R$ will be maximum if $R^2$ is also maximum. So, the optimization problem is,
\begin{maxi}
{x}{2ax^3-x^4}{}{}
\addConstraint{x}{>0}{}
\addConstraint{x}{<2a}{}
\end{maxi}

Using gradient ascent method we can find the maximum of $R^2$, i.e.\ \eqref{eq:fofx}.
\begin{align}
	x_{n+1} &= x_n + \alpha \nabla f(x_n) \nonumber \\
	&= x_n + \alpha \brak{6a{x_n}^2-4{x_n}^3}
\end{align}

Suppose the semi-axes of the ellipse are $a=5$ and $b=3$. Taking $x_0 = 1$, $\alpha=0.0001$ and $\quad$precision $= 0.000000001$, values obtained using python are:
\begin{align}
		\boxed{\text{Maxima Point} = 7.499999956932471} \label{eq:maxPt} \\
		\boxed{\text{Maxima} = 1054.687499999999} \label{eq:maxV}
\end{align}
The corresponding value of $S$ obtained using \eqref{eq:scaleBack} and \eqref{eq:maxV} is, \[ \boxed{S_{opt}=19.48557158514986} \]
The corresponding value of $S$ obtained using \eqref{eq:ar_ellip} is, \[ \boxed{S=19.48557158514987} \]
\[ \implies S_{opt}= S \]
Hence, we have found the maximum area of a triangle that can be inscribed in an ellipse.
%Since, $S_{opt}=S$ it is verified that the greatest triangle inscribed in an ellipse has area of $\cfrac{3\sqrt{3}}{4}{}ab$.
\medskip

\begin{figure}[h]
\centering
\def\figwidth{\linewidth}
\def\figheight{0.315\textheight} % Feel free to change
% This file was created with tikzplotlib v0.10.1.
\begin{tikzpicture}

\definecolor{darkgray176}{RGB}{176,176,176}
\definecolor{darkorange25512714}{RGB}{255,127,14}
\definecolor{steelblue31119180}{RGB}{31,119,180}
\tikzstyle{every node}=[font=\normalsize]

\begin{axis}[
legend style={nodes={scale=0.75,font= \small, transform shape}, at={(1,0)}, anchor=south east, draw=black},
height=\figheight,
tick align=outside,
tick pos=left,
width=\figwidth,
x grid style={darkgray176},
xlabel={\(\displaystyle x\)},
xmajorgrids,
xmin=-2.65, xmax=11.65,
xtick style={color=black},
y grid style={darkgray176},
ylabel style={rotate=-90.0},
ylabel={\(\displaystyle R^2\)},
ymajorgrids,
ymin=-1450.284375, ymax=1173.971875,
ytick style={color=black}
]
\addplot [semithick, steelblue31119180]
table {%
-2 -96
-1.98698698698699 -94.0362068204667
-1.97397397397397 -92.1006249224266
-1.96096096096096 -90.1930173489446
-1.94794794794795 -88.3131478312985
-1.93493493493493 -86.4607807889789
-1.92192192192192 -84.6356813296889
-1.90890890890891 -82.8376152493445
-1.8958958958959 -81.0663490320744
-1.88288288288288 -79.3216498502201
-1.86986986986987 -77.6032855643355
-1.85685685685686 -75.9110247231876
-1.84384384384384 -74.2446365637559
-1.83083083083083 -72.6038910112328
-1.81781781781782 -70.9885586790232
-1.8048048048048 -69.398410868745
-1.79179179179179 -67.8332195702286
-1.77877877877878 -66.292757461517
-1.76576576576577 -64.7767979088664
-1.75275275275275 -63.2851149667453
-1.73973973973974 -61.8174833778352
-1.72672672672673 -60.37367857303
-1.71371371371371 -58.9534766714367
-1.7007007007007 -57.5566544803747
-1.68768768768769 -56.1829894953763
-1.67467467467467 -54.8322599001866
-1.66166166166166 -53.5042445667631
-1.64864864864865 -52.1987230552765
-1.63563563563564 -50.9154756141097
-1.62262262262262 -49.6542831798588
-1.60960960960961 -48.4149273773324
-1.5965965965966 -47.1971905195516
-1.58358358358358 -46.0008556077508
-1.57057057057057 -44.8257063313765
-1.55755755755756 -43.6715270680884
-1.54454454454454 -42.5381028837586
-1.53153153153153 -41.4252195324721
-1.51851851851852 -40.3326634565267
-1.50550550550551 -39.2602217864326
-1.49249249249249 -38.2076823409131
-1.47947947947948 -37.174833626904
-1.46646646646647 -36.1614648395539
-1.45345345345345 -35.167365862224
-1.44044044044044 -34.1923272664885
-1.42742742742743 -33.236140312134
-1.41441441441441 -32.29859694716
-1.4014014014014 -31.3794898077788
-1.38838838838839 -30.4786122184153
-1.37537537537538 -29.5957581917071
-1.36236236236236 -28.7307224285046
-1.34934934934935 -27.8833003178709
-1.33633633633634 -27.0532879370818
-1.32332332332332 -26.2404820516259
-1.31031031031031 -25.4446801152044
-1.2972972972973 -24.6656802697314
-1.28428428428428 -23.9032813453335
-1.27127127127127 -23.1572828603502
-1.25825825825826 -22.4274850213337
-1.24524524524525 -21.7136887230489
-1.23223223223223 -21.0156955484734
-1.21921921921922 -20.3333077687975
-1.20620620620621 -19.6663283434243
-1.19319319319319 -19.0145609199696
-1.18018018018018 -18.3778098342619
-1.16716716716717 -17.7558801103425
-1.15415415415415 -17.1485774604653
-1.14114114114114 -16.555708285097
-1.12812812812813 -15.977079672917
-1.11511511511512 -15.4124994008174
-1.1021021021021 -14.8617759339033
-1.08908908908909 -14.324718425492
-1.07607607607608 -13.8011367171139
-1.06306306306306 -13.2908413385121
-1.05005005005005 -12.7936435076423
-1.03703703703704 -12.309355130673
-1.02402402402402 -11.8377888019854
-1.01101101101101 -11.3787578041735
-0.997997997997998 -10.9320761080439
-0.984984984984985 -10.497558372616
-0.971971971971972 -10.0750199451218
-0.958958958958959 -9.6642768610063
-0.945945945945946 -9.26514584392697
-0.932932932932933 -8.87744430575408
-0.91991991991992 -8.50099034657067
-0.906906906906907 -8.13560275467244
-0.893893893893894 -7.78110100656785
-0.880880880880881 -7.43730526697809
-0.867867867867868 -7.10403638883706
-0.854854854854855 -6.78111591329137
-0.841841841841842 -6.46836606970038
-0.828828828828829 -6.16560977563616
-0.815815815815816 -5.87267063688352
-0.802802802802803 -5.58937294743998
-0.78978978978979 -5.31554168951578
-0.776776776776777 -5.05100253353389
-0.763763763763764 -4.79558183813002
-0.750750750750751 -4.54910665015259
-0.737737737737738 -4.31140470466273
-0.724724724724725 -4.08230442493432
-0.711711711711712 -3.86163492245396
-0.698698698698699 -3.64922599692096
-0.685685685685686 -3.44490813624736
-0.672672672672673 -3.24851251655793
-0.65965965965966 -3.05987100219017
-0.646646646646647 -2.87881614569429
-0.633633633633634 -2.70518118783323
-0.620620620620621 -2.53880005758265
-0.607607607607608 -2.37950737213094
-0.594594594594595 -2.22713843687922
-0.581581581581582 -2.08152924544133
-0.568568568568569 -1.94251647964382
-0.555555555555556 -1.80993750952599
-0.542542542542543 -1.68363039333984
-0.529529529529529 -1.56343387755011
-0.516516516516516 -1.44918739683426
-0.503503503503504 -1.34073107408247
-0.490490490490491 -1.23790572039765
-0.477477477477477 -1.14055283509544
-0.464464464464464 -1.04851460570419
-0.451451451451451 -0.961633907964986
-0.438438438438439 -0.87975430583163
-0.425425425425425 -0.802720051470652
-0.412412412412412 -0.730376085261308
-0.399399399399399 -0.662568035795577
-0.386386386386386 -0.599142219878163
-0.373373373373373 -0.539945642526493
-0.36036036036036 -0.484825996970721
-0.347347347347347 -0.43363166465372
-0.334334334334334 -0.386211715231094
-0.321321321321321 -0.342415906571166
-0.308308308308308 -0.302094684754988
-0.295295295295295 -0.265099184076333
-0.282282282282282 -0.231281227041698
-0.269269269269269 -0.200493324370308
-0.256256256256256 -0.172588674994109
-0.243243243243243 -0.147421166057772
-0.23023023023023 -0.124845372918694
-0.217217217217217 -0.104716559146995
-0.204204204204204 -0.0868906765255188
-0.191191191191191 -0.0712243650498353
-0.178178178178178 -0.0575749529282376
-0.165165165165165 -0.0458004565817435
-0.152152152152152 -0.0357595806440952
-0.139139139139139 -0.0273117179617589
-0.126126126126126 -0.0203169495939256
-0.113113113113113 -0.0146360448125107
-0.1001001001001 -0.0101304611021537
-0.087087087087087 -0.00666234416021861
-0.0740740740740742 -0.00409452789679383
-0.0610610610610611 -0.00229053443469201
-0.0480480480480481 -0.00111457410945033
-0.035035035035035 -0.000431545469330255
-0.022022022022022 -0.000107035275317652
-0.00900900900900892 -7.31850112275073e-06
0.00400400400400391 6.41666819844441e-07
0.0170170170170172 4.91938293511561e-05
0.03003003003003 0.000269998374587827
0.0430430430430429 0.000794027477922126
0.0560560560560561 0.00175156510202199
0.069069069069069 0.00327220699683088
0.0820820820820822 0.00548486069956806
0.0950950950950951 0.00851774553472816
0.108108108108108 0.0124983926140816
0.121121121121121 0.0175536448366747
0.134134134134134 0.0238096568888287
0.147147147147147 0.0313918952441414
0.16016016016016 0.040425138163485
0.173173173173173 0.0510334756950085
0.186186186186186 0.0633403096741364
0.199199199199199 0.0774683537235677
0.212212212212212 0.0935396332532787
0.225225225225225 0.111675485460519
0.238238238238238 0.131996559329818
0.251251251251251 0.154622815632975
0.264264264264264 0.179673526929069
0.277277277277277 0.207267277564454
0.29029029029029 0.237521963672759
0.303303303303303 0.27055479317489
0.316316316316316 0.306482285779024
0.329329329329329 0.34542027298062
0.342342342342342 0.38748389806241
0.355355355355355 0.432787616094398
0.368368368368369 0.481445193933872
0.381381381381381 0.533569710225385
0.394394394394394 0.589273555400775
0.407407407407407 0.648668431679152
0.42042042042042 0.711865353066898
0.433433433433434 0.778974645357679
0.446446446446446 0.850105946132426
0.45945945945946 0.925368204759357
0.472472472472472 1.00486968239396
0.485485485485485 1.08871795197899
0.498498498498499 1.17701989824449
0.511511511511511 1.26988171770778
0.524524524524525 1.36740891867345
0.537537537537538 1.46970632123336
0.55055055055055 1.57687805726666
0.563563563563564 1.68902757043976
0.576576576576576 1.80625761620635
0.58958958958959 1.92867026180741
0.602602602602603 2.05636688627118
0.615615615615615 2.18944818041317
0.628628628628629 2.32801414683619
0.641641641641642 2.4721640999303
0.654654654654655 2.62199666587286
0.667667667667668 2.77760978262848
0.68068068068068 2.93910069994906
0.693693693693694 3.10656597937378
0.706706706706707 3.28010149422907
0.71971971971972 3.45980242962869
0.732732732732733 3.64576328247361
0.745745745745746 3.83807786145212
0.758758758758759 4.03683928703976
0.771771771771772 4.24213999149936
0.784784784784785 4.45407171888105
0.797797797797798 4.67272552502217
0.810810810810811 4.8981917775474
0.823823823823824 5.13056015586865
0.836836836836837 5.36991965118514
0.84984984984985 5.61635856648335
0.862862862862863 5.86996451653703
0.875875875875876 6.13082442790723
0.888888888888889 6.39902453894223
0.901901901901902 6.67465039977763
0.914914914914915 6.9577868723363
0.927927927927928 7.24851813032835
0.940940940940941 7.54692765925122
0.953953953953954 7.85309825638956
0.966966966966967 8.16711203081536
0.97997997997998 8.48905040338784
0.992992992992993 8.81899410675353
1.00600600600601 9.15702318534621
1.01901901901902 9.50321699538694
1.03203203203203 9.85765420488407
1.04504504504505 10.2204127936332
1.05805805805806 10.5915700532172
1.07107107107107 10.9712025870063
1.08408408408408 11.359386310158
1.0970970970971 11.7561964496168
1.11011011011011 12.1617075441149
1.12312312312312 12.5759934441714
1.13613613613614 12.999127312093
1.14914914914915 13.4311816219735
1.16216216216216 13.8722281596939
1.17517517517518 14.3223380229226
1.18818818818819 14.7815816211153
1.2012012012012 15.2500286755149
1.21421421421421 15.7277482191516
1.22722722722723 16.2148085968428
1.24024024024024 16.7112774651934
1.25325325325325 17.2172217925953
1.26626626626627 17.7327078592279
1.27927927927928 18.2578012570576
1.29229229229229 18.7925668898385
1.30530530530531 19.3370689731115
1.31831831831832 19.8913710342052
1.33133133133133 20.4555359122352
1.34434434434434 21.0296257581043
1.35735735735736 21.613702034503
1.37037037037037 22.2078255159086
1.38338338338338 22.812056288586
1.3963963963964 23.4264537505872
1.40940940940941 24.0510766117516
1.42242242242242 24.6859828937057
1.43543543543544 25.3312299298635
1.44844844844845 25.986874365426
1.46146146146146 26.6529721573818
1.47447447447447 27.3295785745066
1.48748748748749 28.0167481973632
1.5005005005005 28.7145349183021
1.51351351351351 29.4229919414607
1.52652652652653 30.1421717827639
1.53953953953954 30.8721262699237
1.55255255255255 31.6129065424395
1.56556556556557 32.364563051598
1.57857857857858 33.127145560473
1.59159159159159 33.9007031439257
1.6046046046046 34.6852841886047
1.61761761761762 35.4809363929457
1.63063063063063 36.2877067671716
1.64364364364364 37.1056416332928
1.65665665665666 37.9347866251069
1.66966966966967 38.7751866881986
1.68268268268268 39.6268860799402
1.6956956956957 40.489928369491
1.70870870870871 41.3643564377976
1.72172172172172 42.2502124775941
1.73473473473473 43.1475379934017
1.74774774774775 44.0563738015287
1.76076076076076 44.9767600300712
1.77377377377377 45.908736118912
1.78678678678679 46.8523408197215
1.7997997997998 47.8076121959573
1.81281281281281 48.7745876228643
1.82582582582583 49.7533037874746
1.83883883883884 50.7437966886077
1.85185185185185 51.7461016368703
1.86486486486486 52.7602532546563
1.87787787787788 53.7862854761471
1.89089089089089 54.8242315473111
1.9039039039039 55.8741240259041
1.91691691691692 56.9359947814692
1.92992992992993 58.0098749953369
1.94294294294294 59.0957951606246
1.95595595595596 60.1937850822374
1.96896896896897 61.3038738768674
1.98198198198198 62.4260899729941
1.99499499499499 63.5604611108841
2.00800800800801 64.7070143425916
2.02102102102102 65.8657760319579
2.03403403403403 67.0367718546114
2.04704704704705 68.220026797968
2.06006006006006 69.4155651612308
2.07307307307307 70.6234105553902
2.08608608608609 71.843585903224
2.0990990990991 73.076113439297
2.11211211211211 74.3210147099613
2.12512512512513 75.5783105733566
2.13813813813814 76.8480211994097
2.15115115115115 78.1301660698344
2.16416416416416 79.4247639781323
2.17717717717718 80.7318330295918
2.19019019019019 82.0513906412888
2.2032032032032 83.3834535420865
2.21621621621622 84.7280377726353
2.22922922922923 86.0851586853729
2.24224224224224 87.4548309445243
2.25525525525526 88.8370685261017
2.26826826826827 90.2318847179046
2.28128128128128 91.6392921195198
2.29429429429429 93.0593026423216
2.30730730730731 94.4919275094711
2.32032032032032 95.9371772559169
2.33333333333333 97.395061728395
2.34634634634635 98.8655900854286
2.35935935935936 100.348770797328
2.37237237237237 101.844611646191
2.38538538538539 103.353119725903
2.3983983983984 104.874301442136
2.41141141141141 106.40816251235
2.42442442442442 107.95470796579
2.43743743743744 109.513942143492
2.45045045045045 111.085868698277
2.46346346346346 112.670490594754
2.47647647647648 114.267810109318
2.48948948948949 115.877828830152
2.5025025025025 117.500547657228
2.51551551551552 119.135966802302
2.52852852852853 120.78408578892
2.54154154154154 122.444903452415
2.55455455455455 124.118417939906
2.56756756756757 125.804626710299
2.58058058058058 127.503526534289
2.59359359359359 129.215113494357
2.60660660660661 130.939382984772
2.61961961961962 132.67632971159
2.63263263263263 134.425947692654
2.64564564564565 136.188230257595
2.65865865865866 137.96317004783
2.67167167167167 139.750759016564
2.68468468468468 141.550988428791
2.6976976976977 143.363848861289
2.71071071071071 145.189330202625
2.72372372372372 147.027421653154
2.73673673673674 148.878111725017
2.74974974974975 150.741388242143
2.76276276276276 152.617238340247
2.77577577577578 154.505648466834
2.78878878878879 156.406604381192
2.8018018018018 158.320091154402
2.81481481481481 160.246093169326
2.82782782782783 162.184594120619
2.84084084084084 164.135577014719
2.85385385385385 166.099024169853
2.86686686686687 168.074917216036
2.87987987987988 170.06323709507
2.89289289289289 172.063964060542
2.90590590590591 174.077077677828
2.91891891891892 176.102556824094
2.93193193193193 178.140379688287
2.94494494494494 180.190523771148
2.95795795795796 182.2529658852
2.97097097097097 184.327682154756
2.98398398398398 186.414648015917
2.996996996997 188.513838216568
3.01001001001001 190.625226816384
3.02302302302302 192.748787186827
3.03603603603604 194.884492011145
3.04904904904905 197.032313284374
3.06206206206206 199.192222313338
3.07507507507508 201.364189716648
3.08808808808809 203.5481854247
3.1011011011011 205.744178679682
3.11411411411411 207.952138035564
3.12712712712713 210.172031358106
3.14014014014014 212.403825824856
3.15315315315315 214.647487925148
3.16616616616617 216.902983460104
3.17917917917918 219.170277542631
3.19219219219219 221.449334597427
3.20520520520521 223.740118360974
3.21821821821822 226.042591881544
3.23123123123123 228.356717519194
3.24424424424424 230.68245694577
3.25725725725726 233.019771144903
3.27027027027027 235.368620412014
3.28328328328328 237.728964354309
3.2962962962963 240.100761890784
3.30930930930931 242.483971252219
3.32232232232232 244.878549981183
3.33533533533534 247.284454932033
3.34834834834835 249.701642270912
3.36136136136136 252.13006747575
3.37437437437437 254.569685336266
3.38738738738739 257.020449953965
3.4004004004004 259.482314742139
3.41341341341341 261.955232425868
3.42642642642643 264.439155042019
3.43943943943944 266.934033939247
3.45245245245245 269.439819777992
3.46546546546547 271.956462530485
3.47847847847848 274.48391148074
3.49149149149149 277.022115224562
3.5045045045045 279.571021669542
3.51751751751752 282.130578035056
3.53053053053053 284.700730852271
3.54354354354354 287.281425964139
3.55655655655656 289.8726085254
3.56956956956957 292.474223002581
3.58258258258258 295.086213173996
3.5955955955956 297.708522129746
3.60860860860861 300.341092271722
3.62162162162162 302.983865313599
3.63463463463463 305.63678228084
3.64764764764765 308.299783510697
3.66066066066066 310.972808652206
3.67367367367367 313.655796666194
3.68668668668669 316.348685825274
3.6996996996997 319.051413713844
3.71271271271271 321.763917228092
3.72572572572573 324.486132575993
3.73873873873874 327.217995277307
3.75175175175175 329.959440163585
3.76476476476476 332.710401378162
3.77777777777778 335.470812376162
3.79079079079079 338.240605924496
3.8038038038038 341.019714101861
3.81681681681682 343.808068298743
3.82982982982983 346.605599217414
3.84284284284284 349.412236871935
3.85585585585586 352.227910588152
3.86886886886887 355.052549003701
3.88188188188188 357.886080068002
3.89489489489489 360.728431042265
3.90790790790791 363.579528499485
3.92092092092092 366.439298324447
3.93393393393393 369.307665713722
3.94694694694695 372.184555175667
3.95995995995996 375.069890530427
3.97297297297297 377.963594909936
3.98598598598599 380.865590757913
3.998998998999 383.775799829865
4.01201201201201 386.694143193087
4.02502502502503 389.620541226661
4.03803803803804 392.554913621455
4.05105105105105 395.497179380126
4.06406406406406 398.447256817118
4.07707707707708 401.405063558661
4.09009009009009 404.370516542773
4.1031031031031 407.343532019261
4.11611611611612 410.324025549715
4.12912912912913 413.311912007517
4.14214214214214 416.307105577834
4.15515515515516 419.30951975762
4.16816816816817 422.319067355616
4.18118118118118 425.335660492353
4.19419419419419 428.359210600146
4.20720720720721 431.389628423098
4.22022022022022 434.426824017101
4.23323323323323 437.470706749833
4.24624624624625 440.521185300759
4.25925925925926 443.578167661133
4.27227227227227 446.641561133993
4.28528528528529 449.711272334167
4.2982982982983 452.78720718827
4.31131131131131 455.869270934703
4.32432432432432 458.957368123656
4.33733733733734 462.051402617105
4.35035035035035 465.151277588814
4.36336336336336 468.256895524332
4.37637637637638 471.368158221
4.38938938938939 474.484966787941
4.4024024024024 477.60722164607
4.41541541541542 480.734822528085
4.42842842842843 483.867668478474
4.44144144144144 487.005657853512
4.45445445445445 490.14868832126
4.46746746746747 493.296656861567
4.48048048048048 496.449459766071
4.49349349349349 499.606992638194
4.50650650650651 502.769150393148
4.51951951951952 505.935827257931
4.53253253253253 509.106916771327
4.54554554554555 512.282311783911
4.55855855855856 515.461904458042
4.57157157157157 518.645586267867
4.58458458458458 521.83324799932
4.5975975975976 525.024779750125
4.61061061061061 528.22007092979
4.62362362362362 531.41901025961
4.63663663663664 534.621485772672
4.64964964964965 537.827384813843
4.66266266266266 541.036594039784
4.67567567567568 544.24899941894
4.68868868868869 547.464486231543
4.7017017017017 550.682939069614
4.71471471471471 553.90424183696
4.72772772772773 557.128277749175
4.74074074074074 560.354929333642
4.75375375375375 563.584078429529
4.76676676676677 566.815606187794
4.77977977977978 570.049393071179
4.79279279279279 573.285318854216
4.80580580580581 576.523262623223
4.81881881881882 579.763102776305
4.83183183183183 583.004717023356
4.84484484484484 586.247982386055
4.85785785785786 589.492775197871
4.87087087087087 592.738971104057
4.88388388388388 595.986445061655
4.8968968968969 599.235071339495
4.90990990990991 602.484723518193
4.92292292292292 605.735274490153
4.93593593593594 608.986596459566
4.94894894894895 612.23856094241
4.96196196196196 615.491038766451
4.97497497497497 618.743900071241
4.98798798798799 621.997014308122
5.001001001001 625.250250240219
5.01401401401401 628.503475942449
5.02702702702703 631.756558801512
5.04004004004004 635.009365515898
5.05305305305305 638.261762095884
5.06606606606607 641.513613863533
5.07907907907908 644.764785452697
5.09209209209209 648.015140809013
5.10510510510511 651.264543189907
5.11811811811812 654.512855164593
5.13113113113113 657.759938614071
5.14414414414414 661.005654731127
5.15715715715716 664.249864020336
5.17017017017017 667.492426298062
5.18318318318318 670.733200692452
5.1961961961962 673.972045643443
5.20920920920921 677.20881890276
5.22222222222222 680.443377533912
5.23523523523524 683.6755779122
5.24824824824825 686.905275724708
5.26126126126126 690.132325970308
5.27427427427427 693.356582959663
5.28728728728729 696.577900315218
5.3003003003003 699.796130971209
5.31331331331331 703.011127173657
5.32632632632633 706.222740480372
5.33933933933934 709.430821760951
5.35235235235235 712.635221196777
5.36536536536537 715.835788281022
5.37837837837838 719.032371818643
5.39139139139139 722.224819926387
5.4044044044044 725.412980032786
5.41741741741742 728.596698878162
5.43043043043043 731.77582251462
5.44344344344344 734.950196306056
5.45645645645646 738.119664928153
5.46946946946947 741.284072368378
5.48248248248248 744.44326192599
5.4954954954955 747.597076212031
5.50850850850851 750.745357149334
5.52152152152152 753.887945972516
5.53453453453453 757.024683227983
5.54754754754755 760.155408773929
5.56056056056056 763.279961780332
5.57357357357357 766.398180728962
5.58658658658659 769.509903413372
5.5995995995996 772.614966938905
5.61261261261261 775.713207722691
5.62562562562563 778.804461493645
5.63863863863864 781.888563292471
5.65165165165165 784.965347471662
5.66466466466466 788.034647695496
5.67767767767768 791.096296940037
5.69069069069069 794.150127493139
5.7037037037037 797.195970954443
5.71671671671672 800.233658235376
5.72972972972973 803.263019559152
5.74274274274274 806.283884460774
5.75575575575576 809.296081787032
5.76876876876877 812.299439696502
5.78178178178178 815.293785659548
5.79479479479479 818.278946458321
5.80780780780781 821.25474818676
5.82082082082082 824.22101625059
5.83383383383383 827.177575367324
5.84684684684685 830.124249566264
5.85985985985986 833.060862188495
5.87287287287287 835.987235886894
5.88588588588589 838.903192626123
5.8988988988989 841.80855368263
5.91191191191191 844.703139644653
5.92492492492492 847.586770412215
5.93793793793794 850.459265197128
5.95095095095095 853.32044252299
5.96396396396396 856.170120225187
5.97697697697698 859.008115450893
5.98998998998999 861.834244659068
6.003003003003 864.648323620458
6.01601601601602 867.450167417599
6.02902902902903 870.239590444814
6.04204204204204 873.016406408211
6.05505505505506 875.780428325688
6.06806806806807 878.531468526928
6.08108108108108 881.269338653402
6.09409409409409 883.993849658371
6.10710710710711 886.704811806878
6.12012012012012 889.402034675758
6.13313313313313 892.08532715363
6.14614614614615 894.754497440902
6.15915915915916 897.40935304977
6.17217217217217 900.049700804214
6.18518518518519 902.675346840007
6.1981981981982 905.286096604702
6.21121121121121 907.881754857645
6.22422422422422 910.462125669968
6.23723723723724 913.027012424588
6.25025025025025 915.576217816211
6.26326326326326 918.109543851331
6.27627627627628 920.626791848228
6.28928928928929 923.12776243697
6.3023023023023 925.612255559412
6.31531531531532 928.080070469195
6.32832832832833 930.53100573175
6.34134134134134 932.964859224293
6.35435435435435 935.381428135829
6.36736736736737 937.780508967148
6.38038038038038 940.161897530829
6.39339339339339 942.525388951239
6.40640640640641 944.87077766453
6.41941941941942 947.197857418643
6.43243243243243 949.506421273306
6.44544544544545 951.796261600033
6.45845845845846 954.067170082127
6.47147147147147 956.318937714677
6.48448448448448 958.551354804561
6.4974974974975 960.764210970441
6.51051051051051 962.957295142771
6.52352352352352 965.130395563787
6.53653653653654 967.283299787517
6.54954954954955 969.415794679774
6.56256256256256 971.527666418157
6.57557557557558 973.618700492055
6.58858858858859 975.688681702643
6.6016016016016 977.737394162883
6.61461461461461 979.764621297525
6.62762762762763 981.770145843105
6.64064064064064 983.753749847948
6.65365365365365 985.715214672165
6.66666666666667 987.654320987654
6.67967967967968 989.570848778103
6.69269269269269 991.464577338983
6.70570570570571 993.335285277555
6.71871871871872 995.182750512868
6.73173173173173 997.006750275757
6.74474474474474 998.807061108842
6.75775775775776 1000.58345886654
6.77077077077077 1002.33571871503
6.78378378378378 1004.06361513232
6.7967967967968 1005.76692190816
6.80980980980981 1007.44541214413
6.82282282282282 1009.09885825356
6.83583583583584 1010.72703196158
6.84884884884885 1012.32970430512
6.86186186186186 1013.9066456329
6.87487487487487 1015.45762560539
6.88788788788789 1016.98241319488
6.9009009009009 1018.48077668545
6.91391391391391 1019.95248367295
6.92692692692693 1021.39730106502
6.93993993993994 1022.8149950811
6.95295295295295 1024.2053312524
6.96596596596597 1025.56807442194
6.97897897897898 1026.90298874449
6.99199199199199 1028.20983768666
7.00500500500501 1029.48838402679
7.01801801801802 1030.73838985505
7.03103103103103 1031.95961657339
7.04404404404404 1033.15182489552
7.05705705705706 1034.31477484696
7.07007007007007 1035.44822576503
7.08308308308308 1036.55193629881
7.0960960960961 1037.62566440918
7.10910910910911 1038.66916736881
7.12212212212212 1039.68220176215
7.13513513513514 1040.66452348544
7.14814814814815 1041.61588774671
7.16116116116116 1042.53604906577
7.17417417417417 1043.42476127423
7.18718718718719 1044.28177751548
7.2002002002002 1045.10685024469
7.21321321321321 1045.89973122882
7.22622622622623 1046.66017154664
7.23923923923924 1047.38792158867
7.25225225225225 1048.08273105725
7.26526526526527 1048.74434896649
7.27827827827828 1049.37252364228
7.29129129129129 1049.96700272231
7.3043043043043 1050.52753315607
7.31731731731732 1051.05386120482
7.33033033033033 1051.54573244159
7.34334334334334 1052.00289175124
7.35635635635636 1052.42508333038
7.36936936936937 1052.81205068742
7.38238238238238 1053.16353664258
7.3953953953954 1053.47928332782
7.40840840840841 1053.75903218693
7.42142142142142 1054.00252397546
7.43443443443443 1054.20949876076
7.44744744744745 1054.37969592198
7.46046046046046 1054.51285415002
7.47347347347347 1054.60871144761
7.48648648648649 1054.66700512923
7.4994994994995 1054.68747182117
7.51251251251251 1054.66984746151
7.52552552552553 1054.6138673001
7.53853853853854 1054.51926589859
7.55155155155155 1054.38577713041
7.56456456456456 1054.21313418079
7.57757757757758 1054.00106954672
7.59059059059059 1053.74931503701
7.6036036036036 1053.45760177224
7.61661661661662 1053.12566018478
7.62962962962963 1052.75322001878
7.64264264264264 1052.34001033019
7.65565565565566 1051.88575948674
7.66866866866867 1051.39019516796
7.68168168168168 1050.85304436514
7.69469469469469 1050.27403338138
7.70770770770771 1049.65288783156
7.72072072072072 1048.98933264235
7.73373373373373 1048.2830920522
7.74674674674675 1047.53388961136
7.75975975975976 1046.74144818186
7.77277277277277 1045.90548993751
7.78578578578579 1045.02573636392
7.7987987987988 1044.10190825848
7.81181181181181 1043.13372573036
7.82482482482482 1042.12090820054
7.83783783783784 1041.06317440177
7.85085085085085 1039.96024237859
7.86386386386386 1038.81182948732
7.87687687687688 1037.61765239609
7.88988988988989 1036.37742708479
7.9029029029029 1035.09086884512
7.91591591591592 1033.75769228055
7.92892892892893 1032.37761130634
7.94194194194194 1030.95033914956
7.95495495495495 1029.47558834903
7.96796796796797 1027.95307075538
7.98098098098098 1026.38249753103
7.99399399399399 1024.76357915018
8.00700700700701 1023.09602539881
8.02002002002002 1021.37954537471
8.03303303303303 1019.61384748743
8.04604604604605 1017.79863945832
8.05905905905906 1015.93362832052
8.07207207207207 1014.01852041896
8.08508508508509 1012.05302141034
8.0980980980981 1010.03683626317
8.11111111111111 1007.96966925774
8.12412412412412 1005.8512239861
8.13713713713714 1003.68120335214
8.15015015015015 1001.45930957149
8.16316316316316 999.185244171588
8.17617617617618 996.858707991661
8.18918918918919 994.479401182715
8.2022022022022 992.04702320755
8.21521521521522 989.561272840751
8.22822822822823 987.021848168686
8.24124124124124 984.428446589518
8.25425425425425 981.780764813189
8.26726726726727 979.078498861437
8.28028028028028 976.321344067779
8.29329329329329 973.508995077527
8.30630630630631 970.641145847773
8.31931931931932 967.717489647403
8.33233233233233 964.737719057084
8.34534534534535 961.701525969275
8.35835835835836 958.608601588218
8.37137137137137 955.458636429949
8.38438438438438 952.251320322285
8.3973973973974 948.986342404831
8.41041041041041 945.663391128981
8.42342342342342 942.282154257918
8.43643643643644 938.842318866607
8.44944944944945 935.343571341808
8.46246246246246 931.785597382059
8.47547547547548 928.168081997693
8.48848848848849 924.490709510824
8.5015015015015 920.753163555361
8.51451451451451 916.955127076993
8.52752752752753 913.0962823332
8.54054054054054 909.176310893248
8.55355355355355 905.19489363819
8.56656656656657 901.151710760868
8.57957957957958 897.04644176591
8.59259259259259 892.878765469733
8.60560560560561 888.648360000536
8.61861861861862 884.354902798312
8.63163163163163 879.998070614837
8.64464464464464 875.577539513675
8.65765765765766 871.092984870183
8.67067067067067 866.544081371492
8.68368368368368 861.930503016532
8.6966966966967 857.251923116019
8.70970970970971 852.508014292453
8.72272272272272 847.69844848012
8.73573573573574 842.822896925097
8.74874874874875 837.881030185245
8.76176176176176 832.872518130218
8.77477477477477 827.797029941451
8.78778778778779 822.654234112167
8.8008008008008 817.443798447381
8.81381381381381 812.16539006389
8.82682682682683 806.818675390282
8.83983983983984 801.403320166927
8.85285285285285 795.918989445991
8.86586586586587 790.36534759142
8.87887887887888 784.742058278949
8.89189189189189 779.048784496104
8.9049049049049 773.285188542189
8.91791791791792 767.450932028306
8.93093093093093 761.54567587734
8.94394394394394 755.56908032396
8.95695695695696 749.520804914627
8.96996996996997 743.400508507587
8.98298298298298 737.207849272873
8.995995995996 730.942484692308
9.00900900900901 724.6040715595
9.02202202202202 718.192265979844
9.03503503503504 711.706723370522
9.04804804804805 705.147098460503
9.06106106106106 698.513045290548
9.07407407407407 691.8042172132
9.08708708708709 685.02026689279
9.1001001001001 678.160846305439
9.11311311311311 671.225606739051
9.12612612612613 664.214198793323
9.13913913913914 657.126272379733
9.15215215215215 649.961476721552
9.16516516516517 642.719460353833
9.17817817817818 635.399871123419
9.19119119119119 628.002356188944
9.2042042042042 620.526562020822
9.21721721721722 612.97213440126
9.23023023023023 605.338718424245
9.24324324324324 597.625958495561
9.25625625625626 589.833498332774
9.26926926926927 581.960980965237
9.28228228228228 574.008048734091
9.2952952952953 565.974343292264
9.30830830830831 557.859505604473
9.32132132132132 549.66317594722
9.33433433433433 541.384993908795
9.34734734734735 533.024598389276
9.36036036036036 524.581627600526
9.37337337337337 516.0557190662
9.38638638638639 507.446509621736
9.3993993993994 498.75363541436
9.41241241241241 489.976731903087
9.42542542542543 481.115433858717
9.43843843843844 472.169375363837
9.45145145145145 463.138189812826
9.46446446446446 454.021509911846
9.47747747747748 444.818967678845
9.49049049049049 435.530194443563
9.5035035035035 426.154820847521
9.51651651651652 416.692476844037
9.52952952952953 407.142791698207
9.54254254254254 397.505393986918
9.55555555555556 387.779911598842
9.56856856856857 377.965971734442
9.58158158158158 368.063200905966
9.59459459459459 358.071224937452
9.60760760760761 347.989668964721
9.62062062062062 337.818157435382
9.63363363363363 327.556314108833
9.64664664664665 317.203762056261
9.65965965965966 306.760123660635
9.67267267267267 296.225020616717
9.68568568568569 285.598073931053
9.6986986986987 274.878903921972
9.71171171171171 264.067130219606
9.72472472472472 253.162371765855
9.73773773773774 242.164246814418
9.75075075075075 231.072372930772
9.76376376376376 219.886366992196
9.77677677677678 208.605845187743
9.78978978978979 197.23042301826
9.8028028028028 185.759715296375
9.81581581581582 174.193336146514
9.82882882882883 162.530899004876
9.84184184184184 150.772016619459
9.85485485485485 138.916301050045
9.86786786786787 126.963363668201
9.88088088088088 114.912815157284
9.89389389389389 102.764265512435
9.90690690690691 90.5173240405857
9.91991991991992 78.1715993604539
9.93293293293293 65.7266994025431
9.94594594594595 53.1822314091496
9.95895895895896 40.5378019343461
9.97197197197197 27.7930168440034
9.98498498498498 14.9474813157758
9.997997997998 2.00079983910291
10.011011011011 -11.0474237847848
10.024024024024 -24.197586442875
10.037037037037 -37.4500857103594
10.0500500500501 -50.8053198506532
10.0630630630631 -64.2636878153771
10.0760760760761 -77.8255892443613
10.0890890890891 -91.4914244656611
10.1021021021021 -105.261594495534
10.1151151151151 -119.136501038451
10.1281281281281 -133.116546487099
10.1411411411411 -147.202133922379
10.1541541541542 -161.3936671134
10.1671671671672 -175.691550517487
10.1801801801802 -190.096189280179
10.1931931931932 -204.607989235225
10.2062062062062 -219.227356904583
10.2192192192192 -233.954699498432
10.2322322322322 -248.79042491516
10.2452452452452 -263.734941741366
10.2582582582583 -278.788659251864
10.2712712712713 -293.951987409679
10.2842842842843 -309.225336866049
10.2972972972973 -324.60911896043
10.3103103103103 -340.103745720482
10.3233233233233 -355.709629862084
10.3363363363363 -371.427184789323
10.3493493493493 -387.256824594502
10.3623623623624 -403.198964058136
10.3753753753754 -419.254018648955
10.3883883883884 -435.422404523899
10.4014014014014 -451.704538528116
10.4144144144144 -468.100838194981
10.4274274274274 -484.611721746065
10.4404404404404 -501.237608091165
10.4534534534535 -517.978916828282
10.4664664664665 -534.836068243631
10.4794794794795 -551.809483311643
10.4924924924925 -568.899583694963
10.5055055055055 -586.106791744443
10.5185185185185 -603.431530499152
10.5315315315315 -620.874223686371
10.5445445445445 -638.435295721591
10.5575575575576 -656.115171708518
10.5705705705706 -673.914277439073
10.5835835835836 -691.83303939339
10.5965965965966 -709.871884739803
10.6096096096096 -728.031241334878
10.6226226226226 -746.311537723383
10.6356356356356 -764.713203138297
10.6486486486486 -783.236667500818
10.6616616616617 -801.882361420352
10.6746746746747 -820.650716194521
10.6876876876877 -839.542163809154
10.7007007007007 -858.557136938301
10.7137137137137 -877.696068944222
10.7267267267267 -896.959393877381
10.7397397397397 -916.347546476472
10.7527527527528 -935.860962168385
10.7657657657658 -955.500077068231
10.7787787787788 -975.265327979332
10.7917917917918 -995.157152393223
10.8048048048048 -1015.17598848965
10.8178178178178 -1035.32227513658
10.8308308308308 -1055.59645189018
10.8438438438438 -1075.99895899484
10.8568568568569 -1096.53023738315
10.8698698698699 -1117.19072867593
10.8828828828829 -1137.9808751822
10.8958958958959 -1158.9011198992
10.9089089089089 -1179.95190651237
10.9219219219219 -1201.13367939539
10.9349349349349 -1222.44688361012
10.9479479479479 -1243.89196490665
10.960960960961 -1265.46936972329
10.973973973974 -1287.17954518654
10.986986986987 -1309.02293911113
11 -1331
};
\addplot [semithick, darkorange25512714, mark=*, mark size=3, mark options={solid}, only marks]
table {%
7.49999995693247 1054.6875
};
\draw (axis cs:7.49999995693247,1054.6875) node[
  scale=0.5,
  anchor=base west,
  text=black,
  rotate=0.0
]{P(7.5000,1054.6875)};
\end{axis}

\end{tikzpicture}

\label{fig:rough}
\end{figure}

\end{document}
